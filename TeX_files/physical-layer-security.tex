\chapter{Physical Layer Security}
The concern of physical security is preventing unauthorised access  to those tangible assets which process, transform, manipulate data and information. In the IT world such assets may be computer terminals, network servers, switches, cabling etc.
Electrical power supplies needed to run the network are also physical assets.
Automated industry may additionally include various machines, device controllers and environment sensors.
Hard-copy or paper documents should not be discounted when considering physical security.

In this section discussion shall focus on the physical layer referred to by the OSI or TCP/IP networking model. This is the physical medium that carries the information, as well as the connection devices.
\section{Physical Access to Hardware}
\section{Remote Intercept of the Physical Layer}
Here we refer explicitly to the detection of information signals outside of the intended information boundary.

Consider an argument occurring in a block of flats. It is likely that the intended information boundaries are the walls of the single flat in which the argument occurs. However, the argument is heard in surrounding flats because the sound vibrations are conducted (although damped) through the walls. Thus information has been received without being "present" in the argument.

In current practise information may transmitted using methods concerning electromagnetic radiation or mechanical vibration - the transfer of energy as a wave of vibrating particles within a medium.
\subsection{Electromagnetic Radiation}
Information can be transmitted as energy in the electromagnetic spectrum which includes:
\begin{itemize}
	\item Visible light;
	\item Infrared light and thermal frequencies;
	\item Ultraviolet and radio frequencies.
\end{itemize}
Electromagnetic radiation my be deliberately employed to transmit without wires over large distances. Radio is a prime example. However most highschool physics textbook explain that an electrical charge travelling with a velocity creates a magnetic field, or a conductor carrying a current creates a magnetic field perpendicular to the direction of current flow. 
The fundamental principle of radio enginering is: "A single unshielded conductor with internal resistance and carrying an alternating current will radiate energy as electromagnetic waves".\footnote{Bailey, D., Practical Radio Engineering and Telemetry for industry, p2} Therefore, if it is possible to detect the radiated energy and information leak may exist. 
\subsection{Mechanical Energy}
Information can be transmitted through a medium as energy characterised by particle vibration travelling as waves through the medium.
Audible sound in air is a well understood example.
Solids, Liquids and gasses all can transmit vibration however the effectiveness is highly specific to precise conditions of the medium. In general terms, a stiffer the medium will absorb less energy and is therefore a better conductor.