\chapter{Introduction}
\section{Purpose of this book}
This book is NOT a complete and organised text on security. This book IS a sampling of guide notes and reference material collected (and still collecting) whilst studying and working in the security domain. This book is not intended as the be-all and end-all of security but more akin to an identifier of key topics for which additional information should be sought elsewhere. This book does intend (eventually) to discuss the kinds of things one should think about when implement security into a system.
\section{What is security}
Most security text books will tell you that 'Security' is about maintaining the three pillars of 'Integrity', 'Availability' and 'Confidentiality'. \footnote{Insert reference for this first line}
While this is true, the implementation of security processes, procedures and tools inline with "good practise" can make a system cumbersome or impractical for the end user. In a similar vein, implementing security has additional costs (money, time etc.) which could, if security is poorly planned, be unnecessarily excessive. Therefore the art of security is finding a solution that not only optimises the three pillars of security but also the end users budget and operability of the system being secured.
\section{What to to expect}
Before getting bogged down in what kind security technologies should considered, the system being secured must first be understood. Subsequent to that, the threats,  probabilities and risks faced the system must be identified and a risk reduction method selected. This will be the subject of the first two chapters. Subsequent chapters will essentially list and briefly describe various methods and tools used in securing a system.